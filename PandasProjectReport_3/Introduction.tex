

    The purpose of this project is to build a sentiment analysis model to detect the sentiment("Positive","Negative","Neutral") of a given sentence. The project is written with python and using
    keras, scikit-learn, tensorflow, pandas, matplotlib as dependencies. The dataset being used for the project is "Sentiment140 dataset with 1.6 million tweets"from Kaggle website, which contains 1.6 million tweets with their sentiment values(0:Negative,2:Neutral,4:Positive)


\subsection*{Report Structure:}
We will explain the project with 8 chapters, as follows:
\vspace{0.4 cm}
\par Sentiment Analysis chapter provides a general information of sentiment analysis process. \par
\par Data Preprocessing chapter provides a general information of cleaning the data \par
Text Preprocessing chapter  provides a general information of cleaning text from undesired structure  \par
Tokenization chapter  provides a general information of tokenization process and label encoding  \par
Word Embeding chapter  provides a general information of creating embedding layer from a pretrained model \par
Model Training chapter provides a general information of creating the model with sequence models \par
Model Evaluation chapter investigates test results of the model \par
Model Testing chapter gives general information about testing the program with inputs given from the user \par