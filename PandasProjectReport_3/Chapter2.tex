

Data preprocessing is a data mining technique that involves transforming raw data into an understandable format. Real-world data is ften incomplete, inconsistent, lacking in certain behaviors or trends, and is likely to contain many errors. Data preprocessing is a proven method of resolving such issues. Data preprocessing prepares raw data for further processing .That's why we preprocessed our data to make it more useful, and we made 2 columns more useful by deleting 4 out of 6 columns in our data.
For Data preprocessing we did the following below:


Since column names are (1,2,3,4,5,6), we rename the columns to proper names. As ('sentiment', 'id', 'date', 'query', "userid", 'text')

\begin{lstlisting}[
                   language = python,
                   xleftmargin = 0.1cm,
                   framexleftmargin = 1em]
df.columns = ['sentiment', 'id', 'date', 'query', 'user_id', 'text']
\end{lstlisting}

Since we are going to train only on text to classify its sentiment. So we can drop the rest of the columns. Because they are unnecessary.

\begin{lstlisting}[
                   language = python,
                   xleftmargin = 0.1cm,
                   framexleftmargin = 1em]
df = df.drop(['id', 'date', 'query', 'user_id'], axis=1)
\end{lstlisting}

In the sentiment column, there are 3 values, 0 indicating that the sentiment is Negative, 4 indicates that sentiment is Positive and 2 indicates that sentiment is Neutral. That's why we change the values(0,4,2) to (Negative, Positive,Neutral)
\begin{lstlisting}[
                   language = python,
                   xleftmargin = 0.1cm,
                   framexleftmargin = 1em]
lab_to_sentiment = {0:"Negative", 2:"Neutral" 4:"Positive"}
def label_decoder(label):
  return lab_to_sentiment[label]
df.sentiment = df.sentiment.apply(lambda x: label_decoder(x))
\end{lstlisting}
